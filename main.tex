\documentclass[bachelor, och, referat]{SCWorks}
% параметр - тип обучения - одно из значений:
%    spec     - специальность
%    bachelor - бакалавриат (по умолчанию)
%    master   - магистратура
% параметр - форма обучения - одно из значений:
%    och   - очное (по умолчанию)
%    zaoch - заочное
% параметр - тип работы - одно из значений:
%    referat    - реферат
%    coursework - курсовая работа (по умолчанию)
%    diploma    - дипломная работа
%    pract      - отчет по практике
%    pract      - отчет о научно-исследовательской работе
%    autoref    - автореферат выпускной работы
%    assignment - задание на выпускную квалификационную работу
%    review     - отзыв руководителя
%    critique   - рецензия на выпускную работу
% параметр - включение шрифта
%    times    - включение шрифта Times New Roman (если установлен)
%               по умолчанию выключен 

\input{preamble/preamble.sty}

\newcommand{\eqdef}{\stackrel {\rm def}{=}}

\newtheorem{lem}{Лемма}

\begin{document}


% Кафедра (в родительном падеже)
%\chair{математической кибернетики и компьютерных наук}

% Тема работы
\title{Готовые лабораторные работы по физике 2 семестр 1 курс.}

% Курс
\course{1}

% Группа
\group{151}

% Факультет (в родительном падеже) (по умолчанию "факультета КНиИТ")
%\department{факультета КНиИТ}

% Специальность/направление код - наименование

\napravlenie{09.03.04 "--- Программная инженерия}

% Фамилия, имя, отчество в родительном падеже
\author{Смирнова Егора Ильича}

% Год выполнения отчета
\date{2024}

\maketitle

% Включение нумерации рисунков, формул и таблиц по разделам
% (по умолчанию - нумерация сквозная)
% (допускается оба вида нумерации)
%\secNumbering

\tableofcontents

% Раздел "Обозначения и сокращения". Может отсутствовать в работе
%\abbreviations
%\begin{description}
%    \item $|A|$  "--- количество элементов в конечном множестве $A$;
%    \item $\det B$  "--- определитель матрицы $B$;
%    \item ИНС "--- Искусственная нейронная сеть;
%    \item FANN "--- Feedforward Artifitial Neural Network
%\end{description}

% Раздел "Определения". Может отсутствовать в работе
%\definitions

% Раздел "Определения, обозначения и сокращения". Может отсутствовать в работе.
% Если присутствует, то заменяет собой разделы "Обозначения и сокращения" и "Определения"
%\defabbr


% Раздел "Введение"
\intro

Здесь вы можете найти готовые лабораторные работы за 2 семестр 1 курса, которые мы выполняли. Возможно вам придется что"=то переделать, но мне кажется пофиксить пару моментов в готовом сильно проще, чем хуярить с нуля (Делать одну сраную лабу 1.5 месяца очень весело, надеюсь вас это не настигнет).

\section{Лабораторная работа <<Поверхностное натяжение>>}

\input{sections/Laba_on_surface_tension}

\section{Лабораторная работа <<Закон Бойля-Мариотта>>}

\input{sections/Laba_on_Boyle-Marriott}

\section{Лабораторная работа <<Определение отношения молярных теплоёмкостей газов адиабатическим методом>>}

\textbf{Принадлежности:} установка для определения отношения молярных теплоемкостей газа адиабатическим методом.

\begin{center}
    \textbf{Оборудование.}
\end{center}

Для проведения опыта служит прибор (рис. \ref{fig:установка 3.}), состоящий из большой стеклянной бутыли A, в горловину которой вставлена герметичная пробка. Через пробку проходит латунная трубка, имеющая два отвода: один a соединен с насосом, служащим для накачивания воздуха в сосуд, другой c – с манометром M. Сверху латунная трубка плотно закрывается резиновой пробкой в.

\begin{figure}[!h]
    \centering
    \includegraphics[width = 0.3\textwidth]{image/image3.png}
    
    \caption{Установка для Определения отношения молярных теплоемкостей газа адиабатическим методом.}
    
    \label{fig:установка 3.}
\end{figure}

Отношение количества теплоты $dQ$, сообщенного системе (телу), к соответствующему повышению температуры $dT$ называют теплоёмкостью:

\begin{equation}
    C_\text{тела} = \frac{d Q}{d T}
    \label{eq:formula_1}
\end{equation}

Наиболее распространенным является определение теплоёмкости как количества теплоты, которое необходимо затратить для изменения температуры тела на один градус.

Теплоёмкость единицы массы вещества называют удельной: 

\begin{equation*}
c = \frac{1}{m} \cdot \frac{d Q}{d T} [\frac{\text{Дж}}{\text{кг} \cdot \text{К}}]
\end{equation*}

Теплоёмкость моля вещества называют молярной:

\begin{equation}
c = \frac{1}{\nu} \cdot \frac{d Q}{d T} [\frac{\text{Дж}}{\text{моль} \cdot \text{К}}]
    \label{eq:formula_2}
\end{equation}

Иногда употребляется внесистемная единица теплоѐмкости: $\frac{\text{кал}}{\text{моль*град}}$. Калория "--- внесистемная единица измерения количества теплоты (1 кал = 4,187 Дж).

Удельная и молярная теплоѐмкости связаны соотношением.

$c = \frac{C}{M}$, где M "--- молярная масса.

Приведенное выше определение теплоёмкости (\ref{eq:formula_1}) не является достаточным, так как количество теплоты $dQ$, сообщаемое телу, зависит от характера процесса, в результате которого система перешла в новое состояние. Другими словами, необходимо еще указать условия, при которых производится передача количества тепла. Эта неопределенность  обусловлена тем, что количество теплоты не является функцией состояния тела в отличие, например, от внутренней энергии

В связи с отмеченной неоднозначностью возможны различные определения теплоёмкости. Так, для термодинамической системы, состояние которой определяется: давлением ($p$), объёмом ($V$) и температурой ($T$), различают теплоёмкости при постоянном объеме $CV$ и постоянном давлении $Cp$. Эти теплоёмкости характеризуются количеством тепла, сообщаемым системе в условиях, когда остается неизменным либо объем, либо давление.

Согласно первому закону термодинамики, выражающему закон сохранения энергии в области тепловых явлений, количество теплоты $dQ$, сообщаемое системе, затрачивается на увеличение внутренней энергии системы $dU$ и на работу $dA$, которую система совершает над внешней средой:

\begin{equation}
    d Q = d U + d A
    \label{eq:formula_3}
\end{equation}

(Более строго, это соотношение записывается в виде $\delta Q = dU + \delta A$, чтобы подчеркнуть то обстоятельство, что $dU$ является полным дифференциалом, поскольку среди величин $Q$, $U$ и $A$ только $U$ является функцией состояния системы.)
	Работа $dA$ в случае отсутствия магнитных и электрических явлений сопровождается исключительно расширением системы, которая находится под действием внешнего давления $p$, и в этом случае $dA = p \cdot dV$ ($dV > 0$). Тогда

\begin{equation}
    d Q = d U + p \cdot d V
    \label{eq:formula_4}
\end{equation}

Если нагревание происходит при постоянном объеме $V = const$ ($dV = 0$), то все тепло тратится на увеличение внутренней энергии:

$d Q = d U$

Тогда из определения молярной теплоёмкости (\ref{eq:formula_2}):


\begin{equation*}
    C_V = \left (\frac{d Q}{d T} \right)_V = \left (\frac{d U_0}{d T} \right )_V
\end{equation*}

где $U_0$ "--- внутренняя энергия одного моля газа. Отсюда для идеального газа можно записать.


\begin{equation}
    C_V = \frac{d U_0}{d T}
    \label{eq:formula_5}
\end{equation}

так как внутренняя энергия является только функцией температуры $U(T)$.

Для изобарического процесса ($p = const$) из (\ref{eq:formula_4}) и (\ref{eq:formula_5}) следует:

\begin{equation}
    C_p = \left (\frac{d Q}{d T} \right )_p = \frac{d U_0}{d T} + p \left (\frac{d V}{d T} \right )_p = C_V + p \left (\frac{d V}{d T} \right)
    \label{eq:formula_6}
\end{equation}

В соответствии с уравнением состояния идеального газа (для одного моля газа $\nu = 1$)

\begin{equation}
    p V = R T
    \label{eq:formula_7}
\end{equation}

При $p = const$

\begin{equation*}
    \left (\frac{d V}{d T} \right )_p = \frac{R}{p}
\end{equation*}

тогда

\begin{equation}
    C_p = C_V + R
    \label{eq:formula_8}
\end{equation}

Теплоёмкость $Cр$ всегда больше теплоёмкости $CV$. Это связано с работой, совершаемой газом при расширении ($p = const$).

При описании некоторых физических процессов приходится иметь дело с отношением теплоемкостей

\begin{equation}
    \gamma = \frac{C_p}{C_V}
    \label{eq:formula_9}
\end{equation}

Одним их таких процессов, играющих важную роль при изучении тепловых явлений, является адиабатический процесс. Для этого процесса характерна теплоизолированность системы от внешней среды, т. е. процесс протекает без теплообмена с внешней средой. Значит, работа, совершаемая системой в этом случае, производится за счет изменения ее внутренней энергии. Из первого закона термодинамики (\ref{eq:formula_3}) для адиабатического процесса ($dQ = 0$) имеем 

\begin{equation}
    d u + p \cdot d v = 0
    \label{eq:formula_10}
\end{equation}

В случае идеального газа (\ref{eq:formula_5}) для одного моля $dU_0 = CV \cdot dT$. Тогда с учетом этого выражение (\ref{eq:formula_10}) примет вид

\begin{equation}
    C_V d T + p \cdot dV = 0
    \label{eq:formula_11}
\end{equation}

Дифференцируя уравнение (\ref{eq:formula_7}), получим:

\begin{equation*}
    p d V + V d p = R d T
\end{equation*}

и выразим из полученного выражения $dT$:

\begin{equation}
    d T = \frac{1}{R} \left (p d V + V d p \right )
    \label{eq:formula_12}
\end{equation}

Подставим (\ref{eq:formula_12}) в (\ref{eq:formula_11}):

\begin{equation*}
    \frac{c_v}{r} \left (p d v + v d p \right ) + p d v = 0
\end{equation*}

Преобразования последнего выражения с учётом (\ref{eq:formula_8}) и (\ref{eq:formula_9}) дают:

\begin{equation*}
    \frac{C_V + R}{R} p d V + \frac{C_V}{R} V d p = 0
\end{equation*}

\begin{equation*}
    \frac{C_p}{C_V} \cdot \frac{d V}{V} + \frac{d p}{p} = 0
\end{equation*}

\begin{equation}
    \gamma \frac{d V}{V} + \frac{d p}{p} = 0
    \label{eq:formula_13}
\end{equation}

Интегрирование (\ref{eq:formula_13}) даёт:

\begin{equation*}
    \gamma \ln(V) + \ln(p) = const
\end{equation*}

откуда следует, что

\begin{equation}
    p V^ \gamma = const
    \label{eq:formula_14}
\end{equation}

С учётом (\ref{eq:formula_7}) уравнение (\ref{eq:formula_14}) примет вид

\begin{equation}
    T \cdot V^(\gamma - 1) = const
    \label{eq:formula_15}
\end{equation}

Соотношения (\ref{eq:formula_14}) и (\ref{eq:formula_15}) называют уравнениями Пуассона.

Молекулярно-кинетическая теория, рассматривая газ как совокупность свободно движущихся частиц, подчиняющихся законам классической механики, позволяет удовлетворительно объяснить некоторые основные свойства реальных газов. В частности, она дает возможность оценить порядок величин термодинамических характеристик $C_p$, $C_V$ и $\gamma$.

Согласно представлениям кинетической теории, молекулы идеального газа не взаимодействуют между собой, внутренняя энергия такого газа не зависит от изменения объема и давления и является только функцией температуры. В силу полной беспорядочности движения считают, что в среднем на каждую степень свободы приходится энергия, равная $\frac{kT}{2}$ , где $k = 1,3807 \cdot 10^{–23} \frac{\text{Дж}}{\text{К}}$ "--- постоянная Больцмана.

Если рассматривать одноатомный газ, каждая частица которого представляется как точечная масса, совершающая поступательное движение и поэтому имеющая три степени свободы (положение ее в пространстве может быть определено тремя независимыми координатами), то средняя кинетическая энергия такой частицы равна $\frac{3}{2}kT$

Внутренняя энергия многоатомных газов складывается из кинетических энергий поступательного и вращательного движения молекул. Применяя и в этом случае положение о равном распределении энергии по степеням свободы, можно подсчитать среднюю кинетическую энергию $E_0$ многоатомной молекулы:

\begin{equation}
    E_0 = \frac{i}{2} k T
    \label{eq:formula_16}
\end{equation}

где i "--- число степеней свободы молекул. Молекулы двухатомного газа обладают тремя поступательными и двумя вращательными степенями свободы. Поэтому средняя кинетическая энергия двухатомной молекулы.

\begin{equation}
    E_0 = \frac{5}{2} k T
    \label{eq:formula_17}
\end{equation}

Внутреннюю энергию $U_0$ одного моля газа найдем, умножив (\ref{eq:formula_16}) на число молекул $NA$ в одном моле ($N_A = 6,022 \cdot 10^23 \text{моль}^(-1)$ "--- число Авогадро):

\begin{equation}
    U_0 = \frac{i}{2} N_A k T = \frac{i}{2} R T
    \label{eq:formula_18}
\end{equation}

Молярную теплоёмкость газа при постоянном объеме получим, продифференцировав (\ref{eq:formula_18}) по температуре:

\begin{equation}
    C_p = \frac{d U_0}{d T} = \frac{i}{R}
    \label{eq:formula_19}
\end{equation}

Используя (\ref{eq:formula_7}) и (\ref{eq:formula_19}), выразим через число степеней свободы молекулы теплоёмкость $Cp$ и отношение теплоемкостей $\gamma$:

\begin{equation*}
    C_p = \frac{i}{2} R + R = \frac{i + 2}{2} R
\end{equation*}

\begin{equation*}
    \gamma = \frac{C_p}{C_V} = \frac{i + 2}{2}
\end{equation*}

В работе требуется найти отношение $\gamma$ теплоемкостей $Cp$ и $CV$ воздуха. Поскольку воздух состоит в основном из смеси двухатомных газов (водорода, кислорода, азота), каждая молекула которых имеет пять степеней свободы, то отношение молярных (или удельных в соответствии с соотношением $c = \frac{C}{M}$) теплоемкостей для воздуха будет $\gamma = 1.40$. Это довольно хорошо согласуется по порядку величины с экспериментальными данными, полученными для чистого воздуха, свободного от $CO_2$ и паров воды при нормальных условиях.

\begin{center}
    \textbf{Вывод рабочей формулы}
\end{center}

Определение численного значения величины $\gamma$ является целью предлагаемой работы. Величину $C_p$ экспериментально можно найти обычным калориметрическим способом, а определение $C_V$ на опыте связано со значительными трудностями. Вследствие этого представляется более удобным, предварительно определив величины $C_p$ и $\gamma$, вычислить $C_V$. Идея опыта заключается в следующем: допустим, что некоторая масса газа $m = const$ (меньшая, чем масса всего воздуха в сосуде) характеризуется начальным состоянием \ref{eq:formula_16}: объемом $V_0$, давлением $p_0 + H$ ($p_0$ "--- атмосферное давление, $H$ "--- разность уровней жидкости в манометре) и температурой $T_0$. Заставим газ быстро расширяться. Кратковременность процесса позволяет считать тепловой обмен с окружающей средой отсутствующим, т. е. процесс быстрого расширения "--- адиабатическим. В конце процесса состояние \ref{eq:formula_17} газа (для той же массы m) будет определяться объемом $V_1 > V_0$, давлением $p_0$ и температурой $T_1 < T_0$. Предоставим газу нагреваться при постоянном объеме до прежней температуры $T_0$, которая равна температуре окружающей среды. В конце этого процесса состояние \ref{eq:formula_18} будет характеризоваться прежним объемом $V_1$, давлением $p_0 + H$′ и температурой $T_0$. При адиабатическом процессе перехода газа из состояния \ref{eq:formula_16} в состояние \ref{eq:formula_17} будет выполняться согласно (\ref{eq:formula_26}) условие

\begin{equation}
    p_0 \cdot V_0^ \gamma = \left (p_0 +H \right ) V_0^ \gamma
    \label{eq:formula_20}
\end{equation}

Состояние \ref{eq:formula_16} и \ref{eq:formula_18} характеризуются одинаковой температурой, следовательно, переход массы газа $m$ из состояния \ref{eq:formula_16} в состояние \ref{eq:formula_18} можно осуществить по изотерме, т. е

\begin{equation}
    \left (p_0 + H \right ) \cdot V_0 = \left (p_0 +H' \right ) V_1
    \label{eq:formula_21}
\end{equation}

Равества (\ref{eq:formula_20}) и (\ref{eq:formula_21}) можно перписать в виде

\begin{equation}
    \left (\frac{V_0}{V_1} \right )^ \gamma = \frac{p_0}{p_0 + H} 
    ~~ \text{и} ~~ 
    \left (\frac{V_0}{V_1} \right )^ \gamma = \frac{\left (p_0 + H' \right )^ \gamma}{\left (p_0 + H \right )^ \gamma}
    \label{eq:formula_22}
\end{equation}

откуда $\frac{p_0}{p_0 + H} = \frac{\left (p_0 + H' \right )^ \gamma}{\left (p_0 + H \right )^ \gamma}$. Логарифмируя это выражение, получим:

\begin{equation}
    \ln \left (\frac{p_0}{p_0 + H} \right ) = \gamma \ln \left (\frac{\left (p_0 + H' \right )}{\left (p_0 + H \right )} \right )
    \label{eq:formula_23}
\end{equation}

Определим из (\ref{eq:formula_23}) $\gamma$:

\begin{equation}
    \gamma = \frac{\ln \left (\frac{p_0}{p_0 + H} \right )}{\ln \left (\frac{p_0 + H'}{p_0 + H} \right )} = \frac{\ln \left (1 - \frac{H}{p_0 + H} \right )}{\ln \left (1 - \frac{H - H'}{p_0 + H} \right )}
    \label{eq:formula_24}
\end{equation}

Разлагая логарифмы в ряды по формуле

\begin{equation}
    \ln \left (1 - x \right ) = -x + \frac{x^2}{2} - \frac{x^3}{3} + \dots
    \label{eq:formula_25}
\end{equation}

и ограничиваясь только первыми членами разложения, найдем приближенно: 

\begin{equation}
    \gamma \approx \frac{\frac{-H}{p_0 + H}}{\frac{-H - H'}{p_0 + H}} = \frac{H}{H - H'}
    \label{eq:formula_26}
\end{equation}


Таким образом, можно для определения величины $\gamma$ ограничиться наблюдением над изменениями давления газа ($H$ и $H'$) по отношению к атмосферному давлению.

\begin{center}
    \textbf{Порядок выполнения работы}
\end{center}

\begin{enumerate}
    \item {Накачать в сосуд $А$ воздух до давления 20--30 см по манометру $М$ и закрыть зажим на резиновом шланге.}
    \item {Через некоторое время, когда разность уровней манометрической жидкости установиться примерно постоянной, произвести отсчет давления по манометру ($H$).}
    \item {Вынуть пробку (в), тем самым давая возможность газу расшириться, и через очень короткий промежуток времени (2--3 с) снова плотно закрыть отверстие. Таким образом, осуществляется адиабатический процесс перехода газа из одного состояния в другое. При открытой пробке в давление газа в баллоне падает до атмосферного $p_0$. Вместе с тем понижается температура газа, так как процесс расширения газа адиабатический.}
    \item {Наблюдать по манометру увеличение давления газа в сосуде, что связано с нагреванием газа за счет теплообмена с окружающей средой. Заметить момент, когда показания манометра достигнут максимума, и произвести отсчет $H'$.}
    \item {Повторить пп. 1--4 не менее 5 раз и записать результаты наблюдений в таблицу (\ref{tab:Laba_3}).}
    \item {Вычислить $\gamma$ по формуле (\ref{eq:formula_26}), используя измеренные величины $H$ и $H'$.}
    \item {Рассчитать среднее значение $\gamma_\text{ср}$ и погрешности измерений (абсолютные $\Delta \gamma$, $\Delta \gamma_\text{ср}$ и относительную $\frac{\Delta \gamma_\text{ср}}{\gamma_\text{ср}}$).

            $\gamma_\text{ср} = 1.2904$

            $\Delta \gamma_\text{ср} = 0.06896$

            $\frac{\Delta \gamma_\text{ср}}{\gamma_\text{ср}} \approx 0.0534408$
        }
    \item {Рассчитать максимальную относительную погрешность метода измерений по формуле

            $\delta = \frac{\Delta H_\text{ср}}{H_\text{ср}} + \frac{\Delta \left (H - H' \right )_\text{ср}}{H_\text{ср} - H'_\text{ср}} \cdot 100\%$

            $H_\text{ср} = 24.96$

            $H'_\text{ср} = 5.6$

            $\Delta H_\text{ср} = 1.672$

            $\delta = \left (\frac{0.2}{24.96} + \frac{0.2}{24.96 - 5.6} \right ) \cdot 100\% \approx 1.83434\%$
        }
    \item {Проверить неравенство$\left (\frac{\Delta \gamma_\text{ср}}{\gamma_\text{ср}} \right ) \leq \delta$.

            $5.34408 \leq 1.83434$

            Не выполняется
        }
    \item {Провести сравнение $\gamma_\text{ср}$ с теоретическим значением $\gamma$ для воздуха. Расхождение значений оценить по формуле $\left (\frac{|\gamma_\text{ср} – \gamma_{теор}|}{\gamma_{теор}} \right ) \cdot 100\%$.

            $\left (\frac{|1.2904 - 1.4|}{1.4} \right ) \cdot 100\% \approx 7.828571\%$
        }
    \item {Записать результат в виде $\gamma = \gamma_\text{ср} \pm \Delta \gamma_\text{ср}$.

            $\gamma = 1.2904 \pm 0.06896$
        }
\end{enumerate}

\begin{table}[h!]
    \centering
    \begin{tabular}{|c|c|c|c|c|c|c|c|}
        \hline
        \makecell{№} & 
        \makecell{$H$} &
        \makecell{$H'$} &
        \makecell{$\gamma$} &
        \makecell{$\gamma_\text{ср}$} &
        \makecell{$\Delta \gamma$} &
        \makecell{$\Delta \gamma_\text{ср}$} &
        \makecell{$\frac{\Delta \gamma_\text{ср}}{\gamma_\text{ср}} \cdot 100\%$} \\
        \hline
        $1$ & $24$ & $5.8$ & $1.318$ & & $0.0276$ &  &  \\
        \cline{1-4} \cline{6-6} 
        $2$ & $27.8$ & $6$ & $1.275$ & & $0.0154$ & & \\
        \cline{1-4} \cline{6-6} 
        $3$ & $24.8$ & $5.4$ & $1.278$ & $1.2904$ & $0.0124$ & $0.06896$ & $\approx 5.344\%$ \\
        \cline{1-4} \cline{6-6} 
        $4$ & $26.3$ & $5.6$ & $1.270$ & & $0.0204$ & & \\
        \cline{1-4} \cline{6-6} 
        $5$ & $21.9$ & $5.2$ & $1.311$ & & $0.0206$ & & \\
        \hline
    \end{tabular}
    \caption{Эксперементальные данные}
    \label{tab:Laba_3}
\end{table}

\textbf{Вывод 1:}

Найдено отношение молярных теплоёмкостей воздуха адиабатическим методом. Полученное значение отношения молярных теплоёмкостей отличается от теоретического $1.4$ на $7.8\%$ при флуктуации $0.06896$. Погрешность метода составляет $1.83434\%$. При данной погрешности метода получилось заметное отклонение для $\frac{\Delta \gamma_\text{ср}}{\gamma_\text{ср}}$.

\textbf{Вывод 2:}

Получили отношение молярных теплоёмкостей (далее $\gamma$) воздуха методом с адиабатическим процессом. Получено значение $\gamma$ равное $1.2904 \pm 0.06896$. Данное значение имеет отклонение от $\gamma_\text{теор}$ приблизительно в $7,828571\%$. 

Абсолютное отклонение составляет $0,06896$. 

Погрешность метода составила $1,83434$, несоответствие полученных данных с точки зрения теоретических данных может быть объяснено некоторыми особенностями прибора: неисправность трубки, ввиду которой происходила утечка воздуха, приводила к неточным замерам, быстрый поток воздуха из бутыли, из"=за которого необходимо было с сильной осторожностью открывать пробку и закрывать обратно с большой скоростью. Данные факторы могли повлиять на полученные данные и на их расхождение с теоретическими.


















\end{document}
